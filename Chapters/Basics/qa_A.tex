\begin{enumerate}
\item Name advantages and disadvantages of XML
\subitem Advantages: Check auf Wohlgeformtheit und Validierung. Über Applikationsgrenzen hinweg benutzbar, Easy to define, human readable, easy parsing, data-centric AND document-centric usage. Disadvantages: Large overhead in comparison to JSON
\item What is a markup language in general ?
\subitem Bietet eine Syntax, um Metadaten vom tatsächlichen Inhalt zu trennen.
\item Why is metadata so important that it could motivate the design and specification of a new markup language ?
\subitem It is important, because it contains Information about the Information held in the document.
\item Explain the difference between data and document-centric XML
\subitem Data-centric means that it is well structured and has many small data items (Example: DB). Document-centric means that it contains large amounts of text and there are unstructured elements (Example: XHTML)
\item Name five different application fields of XML in practice
\subitem \begin{itemize}
\item Config-Files
\item Layouting
\item Logging
\item WSDL
\item SOAP
\item Webservices
\item Data-Exchange (B2B, B2C)
\end{itemize}
\item In which field does XML apparently have the biggest success ?
\subitem Business to Business and Business to Customer
\item What are XML-RPC and SOAP, and how do they distinguish ?
\subitem Remote Procedure Call: A Middleware packs and extracts the Method Call with its parameters into XML, sends it to the server and extracts it from XML when it calls the service method. You can say that in the case of an RPC call whether the client nor the server sees an XML document (only the middleware).

In case of a SOAP-message (Called SOAP-Envelope) the server as well as the client interacts directly via XML-document.
\item Which other XML concept is used for web services ?
\subitem WSDL: Describes the functionality / ownership / location of a web service.
\end{enumerate}