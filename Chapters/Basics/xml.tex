\begin{itemize}
\item 1996 als Subset von SGML von W3C publiziert
\item XML spezifiziert nur die Regeln für das Hinzufügen von Metadaten.
\item Anders als SGML spezifiziert XML \textbf{nicht}, welche Metadaten erlaubt sind.
\end{itemize}

Creators of XML: The power of metadata warrants bigger files (... terseness is not an aim ...).
Es gibt verschiedene Möglichkeiten um die Grösse der Files zu reduzieren (z.B. Komprimierung).
Das bedeutet allerdings einen Austausch gegen Lesbarkeit und "`Ease of use"'.
Warum soll man nicht komprimierte Files speichern (z.B. ZIP)? Ein Bitfehler kann die ganze Scheisse zerstören.