Ein XML ist "`wohlgeformt"' (well-formed), wenn es die obigen Regeln erfüllt. Trotzdem interpretieren Browser auch nicht-wohlgeformten HTML Code und versuchen zu interpretieren, was der Programmierer gemeint hat.

XML an sich allerdings wäre genau so strikt, wie jede andere Programmiersprache. Ein XML Prozessor wird bei in der Spezifikation als "`fatal"' definierten Errors die Ausführung stoppen (Ein Browser ist \textbf{kein} XML Prozessor, er lässt also teilweise auch richtig übles Zeug zu).