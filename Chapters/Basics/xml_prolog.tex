The prolog is optional, but if it does exist then it must come first.
\begin{lstlisting}[language=XML]
<?xml version="1.0" ?>
<?xml version="1.0" encoding="ISO-8859-1" ?>
<?xml version="1.0" encoding="UTF-16" ?>
<?xml version="1.0" encoding="EUC-JP" standalone="yes" ?>
\end{lstlisting}

Version is always either 1.0 or 1.1.

XML 1.1 is not very widely implemented and is recommended for use only by those who need its unique features.)

XML allows the use of any Unicode subset. But encodings other than UTF-8 and UTF-16 will not necessarily be recognized by every parser. 

Chicken-and-Egg Problem: If an XML processor does not know the encoding, how can it read that my document uses e.g. UTF-16?

Standalone applies only to documents that specify a DTD.