\chapter{Formatting Objects}
XSL-FO documents contain two required sections:\\
The first section details a list of named page layouts to determine the properties of the pages.\\
The second section is a list of document data with markup, that uses the page layouts to determine how the content fills the pages.\\
The data portion is broken up into a sequence of flows, where each flow is attached to a page region. The flows contain a list of blocks\\
which in turn contain a list of text data, inline markup elements, or a combination of both.\\
Content may also be added to the margins of the document, for page numbers, chapter headings and the like.\\



\section{Kontrollfragen}
\begin{enumerate}
\item What is the purpose of XSL-FO ?\\
\textit{Generiert Outputformate (printmedien), ist ein Zwischen format.}
\textit{Am weitesten Verbreitet ist FO im Publishingbereich. Über ein FOP wird ein Endformat aus einem XML generiert.
}


\item Explain the 2-phase process model.\\
\textit{XML wird mit XSL zu FO verarbeitet und FO mittels FOP zu PDF (oder anderes Endformat)}

\item Name typical output formats for FO documents.\\
\textit{PDF, PCL, PS, TeX, RTF}

\item Why do you think did the designers of XSL-FO choose a 2-phase model here.\\
\textit{Die erste Phase trennt das Layout vom Inhalt und die zweite Phase den Inhalt vom Format}


\item Describe the general build-up of FO documents, i.e. which components are required to be specified ?\\
\textit{Wurzelknoten und zwei Sections (Inhalt in Blöcke zusammenzufassen und Layout)}

\item How does the FO processor transform an FO document into the requested output document.\\
\textit{Generiert Blöcke und lässt dann diese automatisch auffüllen}

\end{enumerate}
