\chapter{Processing XML with Java \& Stuff}
\section{Control Question}
\begin{enumerate}
\item XML documents can be processed sequentially or by creating an in-memory representation. Name advantages, disadvantages and a framework implementation of each variant\\

\begin{tabular}{|p{0.12\linewidth}|p{0.3\linewidth}|p{0.3\linewidth}|p{0.12\linewidth}|}
\hline
Technology & Advantages & Disadvantages & Framework\\
\hline
Sequential & Wenn Zeilennummer ausgegeben werden muss, Memorysparend, bei sehr grossen Files & Schwierig zu programmieren, nicht intuitiv & SAX\\
\hline
In-Memory & Gute OO-Unterstützung, XPath kann verwendet werden & Braucht ca. 4-fachen Speicher des Dokuments & LINQ, JDOM\\
\hline
\end{tabular}
\item What is the different between pull and push processing ?\\
\begin{itemize}
\item push: Eventbasiert, Framework pusht in eigenen Eventhandler
\item pull: Man "`holt"' sich die Daten aus dem Dokument, durchläuft es selbständig.
\end{itemize}
\item Explain the steps for processing an XML document with SAX.
\begin{enumerate}
\item Create a parser
\item Give DefaultHandler extension to parser
\item Start reader on XMLFile
\end{enumerate}
\item What is the difference between DOM, JDOM and XDOM ?\\
DOM: Ursprüngliche Implementierung\\
JDOM: Java Implementation, hat NICHTS mit DOM zu tun, ausser, dass es auch eine In-Memory Darstellung ist.\\
XDOM: .NET Implementation
\item Explain the steps for processing an XML document with JDOM.\\
\begin{enumerate}
\item XML Element in Parser füttern
\item Parser spuckt Dokument aus
\item Dokument kann traversiert werden
\end{enumerate}
\item Explain the steps for invoking an XSL transformation from Java.\\
\begin{enumerate}
\item XML Dok laden
\item XSL Dok laden
\item spezifizieren, wohin das Resultatdokument geschrieben wird.
\end{enumerate}
\end{enumerate}

