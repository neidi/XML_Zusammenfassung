\chapter{SVG}
\section{Theory summary}
SVG is an XML format to define vector graphics. But you can also have the best of both worlds, SVG allows to embed bitmap images.\\
SVG is widely used to define pictures and animations. It recently gained importance when it became part of the Open Web Platform\\
introduced with HTML 5.
Webbrowsers display SVG natively.\\
SVG graphics are most often created using graphical editors such as Inkscape for example. You can even create SVG graphics in your\\
browser – try here.
Alternatively, we may directly manipulate an SVG DOM tree from our programs or, as in this course, produce SVG using XSLT.\\
\section{Control Questions}

\begin{enumerate}
\item Explain the difference between a bitmap and a vector image.\\
\textit{bitmap wird f\"{u}r Fotos verwendet $\rightarrow$ Ansammlung von Pixel und vector image f\"{u}r Grafiken und Diagramme, sie setzen sich aus einer Vektroren Beschreibung zusammen.}

\item Name advantages and disadvantages of each format.\\
\textit{Vektor ist kleiner und Skalierbar da es anhand der Beschreibung neu gezeichnet wird, Bitmap hat Fotografische Qualität, leichter Portierbar, weniger Träge von Maleware}

\item Explain how SVG paths work.\\
\textit{Eine Sequenz von Knoten mit Verbindungslinien dazwischen mit relativen und Absoluten Pfaden (grosse und Kleine buchstaben)}

\item What is the purpose and practical importance of XLink ?\\
\textit{wird nur in SVG, MathML ben\"{o}tigt f\"{u}r verlinkung}


\item Where is XLink used in SVG ?\\
\textit{Um definierte Grafiken zu referenzieren und an anderen orten wieder zu verwenden.}

\item Name different methods of how SVG documents are created and manipulated in practice.\\
\textit{Fragen k\"{o}nnne selber geparst werden}

\item How can we equip SVG graphics with animations ?\\
\textit{smil}
\end{enumerate}

\end{enumerate}
