\chapter{XQueries}
\section{Native XML Databases}
A Native XML Database is a computer system that holds the following properties:
\begin{itemize}
  \item It is a database, i.e. a software component integrating both (1) a structured collection of data and (2) access programs to define, manipulate and query that data
  \item Data are stored in documents. It is therefore a special kind of document store.
  \item Data in the documents comply with the XML standard.
  \item XML technologies such as XPath, Xquery or XSL/T can be applied for querying and manipulating data in the database.
\end{itemize}


\section{XML-DB vs SQL-DB}

\begin{table}[htbp]
\centering
\caption{Comparison XML vs. SQL}
\begin{tabular}{ll}
\textbf{XML DB}         & \textbf{SQL DB}        \\ \cline{1-2}
\multicolumn{1}{|l|}{Hierarchical} & \multicolumn{1}{l|}{Relational} \\ \cline{1-2}
\multicolumn{1}{|l|}{Fast access of nested structures} & \multicolumn{1}{l|}{Slow access of nested structures (join)} \\ \cline{1-2}
\multicolumn{1}{|l|}{Works well for aggregating data (1:n)} & \multicolumn{1}{l|}{Works well for interrelated data (n:m) } \\ \cline{1-2}
\multicolumn{1}{|l|}{Working with files and XML-documents directly } & \multicolumn{1}{l|}{Working with abstract tables and tuples} \\ \cline{1-2}
\multicolumn{1}{|l|}{Focus on structure (XSD, DTD)} & \multicolumn{1}{l|}{Focus on consistency (ACID)} \\ \cline{1-2}
\multicolumn{1}{|l|}{Ideal for web applications} & \multicolumn{1}{l|}{Ideal for banking applications} \\ \cline{1-2}

\end{tabular}
\end{table}

\section{XQuery}
XQuery is a language for searching and manipulating data in XML Trees (XDM)\\
XQuery expressions can access data over multiple documents and databases, and extract results very efficiently.\\
XQuery is a FLWOR-expression (pronounced flower), with main syntax elements:\\FOR-LET-WHERE-ORDER-RETURN\\
\\
XQuery is a true superset of XPath, meaning that every single valid
XPath expression is also a valid XQuery expression.\\
Xquery is an extension to Xpath, allowing for nested loops with SQL- like elements such as where statements and order-by clauses.\\
XPath lets you find XML elements by path expressions.\\
XQuery, however, adds a couple of additional powerful constructs that allow you to do much more and let you manipulate or construct entire XML documents on the fly.\\
  => return clause is essential for this feature, where new and arbitrary XML can be added and combined with results from the query

\section{FLWOR and examples}

\begin{itemize}
  \item for: define variables (start name with \$) as FLWOR-Expression
(normally just an Xpath-expression)
  \item let: new variables for some (aggregated) subsets of the for-variables
  \item where: some restrictions on/ between the for-variables and XPath- expressions starting from the for-variables
  \item order: some variable with Xpath-expression (must be a singleton!)
  \item return: a valid XML-structure constructed of constants and Xpath- expressions starting from variables. If constants and expressions are mixed: expressions must be set in curly brackets ({}) – otherwise they are not evaluated, but also taken as (constant) strings!!
\end{itemize}
Beispiel 1:
\begin{lstlisting}[language=XML]
for $v in //Vorlesung
let $orderVar := $v/SWS
where $v/../..[Name="Sokrates"]
order by $orderVar descending
return <Vorlesung>{$v/Titel}{$v/SWS}</Vorlesung>
\end{lstlisting}
returns:
\begin{lstlisting}[language=XML]
<Vorlesung><Titel>Ethik</Titel> <SWS>4</SWS></Vorlesung>
<Vorlesung><Titel>Logik</Titel><SWS>4</SWS></Vorlesung>
<Vorlesung><Titel>Maeeutik</Titel><SWS>2</SWS></Vorlesung>
\end{lstlisting}
Beispiel 2:
\begin{lstlisting}[language=XML]
  for $p in //ProfessorIn
  let $l := sum($p/Vorlesungen/Vorlesung/SWS)
  where $l > 0 order by $l
  return  <ProfessorIn>{$p/Name} <Lehre> {$l} </Lehre></ProfessorIn>
\end{lstlisting}
returns:
\begin{lstlisting}[language=XML]
For each ProfessorIn-Element the name and the sum of their teaching- hours (Semester-Wochen-Stunden).
ProfessorIn-Elements that do not teach are excluded. Teaching-hours is the order-criteria.
\end{lstlisting}

\section{Exercise 9}
List all countries whose name starts with P and which have a population greater than 10
Million, using the following XQuery:
\begin{lstlisting}[language=XML]
  for $country in //country[matches(@name,'^P','i')and
  @population>10000000]
  let $province := count($country/province)return
  <country><name>{$country/@name}</name><provinces>{$province}
  </provinces></country>
\end{lstlisting}
Returns:
\begin{lstlisting}[language=XML]
  <country>
    <name name="Poland"/>
    <provinces>49</provinces>
  </country>
  <country>
    <name name="Pakistan"/>
    <provinces>0</provinces>
  </country>
  <country>
    <name name="Philippines"/>
    <provinces>0</provinces>
  </country>
  <country>
    <name name="Peru"/>
    <provinces>25</provinces>
  </country>
\end{lstlisting}
How many such countries do exist? What is the corresponding XQuery (or XPath) Expression?
\begin{lstlisting}[language=XML]
  for $counting in count(//country[matches(@name,'^P','i')and @population>10000000])
  return $counting
\end{lstlisting}
