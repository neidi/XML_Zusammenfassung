\chapter{XSLT}
\section{Intro}
XSLT transforms documents from a source format to a target format. The source format must be an XML language, while the target format can be anything
However, in most cases the transformation is between XML languages. Typical target languages are XHTML, SVG or FO\\
XSLT is a declarative-functional programming language, i.e. you specify the result rather than how it is obtained.
\section{XSLT Basics and Conflicts}
XSLT-Stylesheets usually contains of templates. Each template specifies an XPath-like expression.
The following example has two different templates, one matching the root element ('/') ond one matching the movie nodes ('movie').
\begin{lstlisting}[language=XML]
<?xml version="1.0" ?>
<xsl:stylesheet version="1.0" xmlns:xsl="http://www.w3.org/1999/XSL/Transform">
  <xsl:template match="/">
    <html>
      <body>
<h1>This is a plain list of James Bond movie titles</h1> <ul>
          <xsl:apply-templates />
        </ul>
      </body>
</html>
  </xsl:template>
  <xsl:template match="movie">
<li>
      <xsl:value-of select="title"/>
</li>
  </xsl:template>
</xsl:stylesheet>
\end{lstlisting}
The XSLT-processort starts from the root of the source tree and finds the template which matches best. If more than one template matches,
a conflict resolution algorithm is called to determine the right template. Since the results vary from implementation to implementation,
ambigious templates should be avoided.\\
Rule:\\
 \green{\textbf{It is an error if this leaves more than one matching template rule.
An XSLT processor may signal the error; if it does not signal the error, it
must recover by choosing, from amongst the matching template rules that are left, the one that occurs last in the stylesheet.}}
\\
Rule:\\
\green{\textbf{More precise (restricting) templates are always executed first.}}

\section{Defining templates}
The first template should always start at the root element.
Each template has its context and all data requests are executed with respect to the context. Here, the context is \code{<movie>}
\begin{lstlisting}[language=XML]
<xsl:template match="movie">
  <li>
    <xsl:value-of select="title"/>
  </li>
</xsl:template>
\end{lstlisting}
Delegation to other template can be done using the \code{<apply-templates>} element which uses a select attribute
 to choose the nodes to process. The default value is the XPath expression node(). \\
 \begin{lstlisting}[language=XML]
<?xml version="1.0" ?>
<xsl:stylesheet version="1.0" xmlns:xsl="http://www.w3.org/1999/XSL/Transform">
<xsl:template match="/">
  <html>
  <h1>This is a plain list of James Bond movie titles</h1> <ul>
    <xsl:apply-templates select="bond_movies/movie/title" /> </ul>
  </html>
</xsl:template>
<xsl:template match="title">
  <li>
    <xsl:value-of select="text()"/>
  </li>
</xsl:template>
</xsl:stylesheet>
\end{lstlisting}
\section{Push vs. Pull processing}
Using a for-each loop as in the example below, is called pull-processing where using templates is called push-processing.
Elements can be pushed to a template or pulled to be used immediately. Templates are more like methods in other programming languages.
\begin{lstlisting}[language=XML]
<xsl:for-each select="bond_movies/movie">
    <li>
      <xsl:value-of select="title"/>
    </li>
</xsl:for-each>
\end{lstlisting}
Templates are better because:
\begin{itemize}
  \item Templates are more flexible and easier to maintain.
  \item Templates can be shared between stylesheets (see later).
  \item Templates can be overwritten (see later).
  \item Templates enable modular code.
\end{itemize}

\section {Conditional Logic}
There is an if statement (without else) which can be used as follows:
\begin{lstlisting}[language=XML]
<xsl:template match="movie">
  <xsl:if test="duration &gt; 130">
    <li style="color:red"> <xsl:value-of select="title"/>
    </li>
  </xsl:if>
  <xsl:if test="duration &lt;= 130">
    <li>
      <xsl:value-of select="title"/>
    </li>
  </xsl:if>
</xsl:template>
\end{lstlisting}
And there's the choose-statement for multiple conditions:
\begin{lstlisting}[language=XML]
<xsl:template match="movie">
  <xsl:choose>
    <xsl:when test="duration &gt; 140">
      <li style="color:red"><xsl:value-of select="title"/></li>
    </xsl:when>
    <xsl:when test="duration &gt; 120">
      <li style="color:orange"><xsl:value-of select="title"/></li>
    </xsl:when>
    <xsl:otherwise>
      <li style="color:green"><xsl:value-of select="title"/></li>
    </xsl:otherwise>
  </xsl:choose>
</xsl:template>
\end{lstlisting}

\section{Producing XHTML}
XHTML requires a default namespace in the <html> element. This can be achieved by adding the XHTML namespace as
default namespace to the XSLT file. It will then be carried over to the output document. Like this:
\begin{lstlisting}[language=XML]
<xsl:stylesheet version="1.0" xmlns:xsl="http://www.w3.org/1999/XSL/Transform" xmlns="http://www.w3.org/1999/xhtml" >
\end{lstlisting}
Because the XHTML-Namespace has to be the default namespace, all the other namespaces have to be imported using prefixes.\\
XHTML is still validated against DTDs, so we may want to refer to the right DTD in our output document. Like this:
\begin{lstlisting}[language=XML]
<xsl:output doctype-public="-//W3C//DTD XHTML 1.0 Strict//EN" doctype-system="http://www.w3.org/TR/xhtml1/DTD/xhtml1-strict.dtd" />
\end{lstlisting}


\section{Sorting elements}
<sort> can be used as a child of <apply-templates> or <for-each>.
It has three attributes: select, data-type and order. The possible values for order are ascending or descending. Here's an example:
\begin{lstlisting}[language=XML]
<xsl:apply-templates select="movie[starts-with(bond/text(), 'Pierce')]">
  <xsl:sort select="bond_girl" data-type="text" order="descending"/>
</xsl:apply-templates>
\end{lstlisting}

\section{Named templates}
Instead of (or in addition to) using the match attribute we can call templates by name. This has the following advantages:\\
1. We can split up large templates and reuse elsewhere\\
2. We can pass parameters to the template. Inside the template,
parameter values are accessed by \code{\$para\_name} like this:
\begin{lstlisting}[language=XML]
<xsl:template name="header">
  <xsl:param name="color" />
  <tr style="background-color:{$color}">
      <th>Title</th>
      <th>Bond Actor</th>
      <th>Bond Girl</th>
      <th>Producer</th>
      <th>Year</th>
      <th>Length</th>
  </tr>
</xsl:template>
\end{lstlisting}
And then be called using the name with a parameter: (name of the parameters are global)
\begin{lstlisting}[language=XML]
<xsl:call-template name="header">
  <xsl:with-param name="color">#990066</xsl:with-param>
</xsl:call-template>
\end{lstlisting}
\red{\textbf{Use named templates if parameter values remain constant, use matching when you need to use data out of the XML.}}

\section{More XSLT features}
\begin{itemize}
  \item Variables can be defined as follows:\\
  \code{<xsl:variable name="version" >1.0beta</xsl:variable>}\\
  However, variables can be initialized only once in their lifetime.
  \item \code{<copy-of select="path" />}\\ Copies content from the source to the output tree,
   i.e. it copies the specified node with its children and attributes.
   This is very handy if you want to transform a document into a modified form of itself.
   \item \code{<import href="path" />}\\ Thi can be used to include another stylesheet by just copy- pasting its content to the current file.
    This enables to build modules of reusable code. In case of conflicts, the imported templates obtain lower priority.
    \item Getting Content from other XML Files:\\
    We can access information in other XML files using the function document(url). The specification allows any URL but processors
generally support only paths (see below), HTTP and HTTPS.
In bond\_movies.xml each movie has an attribute id. Imagine now another file bond\_movies\_media.xml that stores poster images for each movie referenced by ID.
We can get the poster for ID \_02 as follows:
\begin{lstlisting}[language=XML]
  <img src="{document('bond_movies_media.xml')/ bond_movies/movie[@number='_02']/poster/@href}"
  alt="{document('bond_movies_media.xml')/ bond_movies/movie[@number='_02']/title}"/>
\end{lstlisting}
\end{itemize}

\section{XSLT Bootcamp Übung (12 Punkte)}
\input{xslt_bootcamp}